\chapter{基本语法}
\label{chap:grammar}


\section{基本规范}

\textbf{Java区分大小写}

Java采用骆驼命名法,CamelCase。

\section{数据类型}

Java包含8种基本类型。\par
\begin{itemize}
        \item   4种整型
        \item   2种浮点类型 
        \item   1种表示Unicode编码的字符单元的字符类型char
        \item   1种表示真值的boolean类型
\end{itemize}


\subsection{整型}

\renewcommand\arraystretch{2}
\begin{tabular}{l|l|l}
    类型        &      存储需求        &     取值范围        \\               \hline
    byte       &       1字节          & -128 $\sim$ 127     \\
    short      &       2字节          & -32768 $\sim$ 32767  \\
    int        &       4字节          &  -2147483648 $\sim$ 2147483647  \\
    long       &       8字节          &  -9223372036854775808 $\sim$ 9223372036854775807  \\
\end{tabular}\newline

\notebox{Java 没有任何无符号(unsigned)形式的整型。}

长整型值有一个后缀l或者L。

\begin{lstlisting}[style=cjava]
        long number = 123L;             // number值为:123
\end{lstlisting}


\begin{itemize}
    \item   二进制带前缀0b或者0B (从java 7开始)
    \item   八进制带前缀0
    \item   十六进制带前缀0x或者0X
\end{itemize}


\begin{lstlisting}[style=cjava]
        int binaryNumber      = 0B101;              // 二进制表示   5 

        int hexadecimalNumber = 0x123;              // 十六进制表示 291   

        int octonaryNumber    = 0123;              // 八进制表示   83
\end{lstlisting}



同样从Java7开始为数字加下划线,Java编译器会去除这些下划线

\begin{lstlisting}[language=java]
        Sytem.out.println(0_1_0);       // 010表示八进制,输出8
\end{lstlisting}


toBinaryString


\cautionbox{
注意JS INT类型的值范围:
\begin{itemize}
        \item   \href{http://speakingjs.com/es5/ch11.html}{Integers in JavaScript} 
        \item   \href{http://speakingjs.com/es5/ch11.html\#safe\_integers}{Safe Integers}
        \end{itemize}
}


\subsection{浮点数}


\renewcommand\arraystretch{2}
\begin{tabular}{l|l|l}
    类型         &      存储需求        &     取值范围        \\               \hline
    float        &       4字节          & 有效位数为6-7位      \\
    double      &       8字节          & 有效位数为15位       \\
\end{tabular}\newline


float类型的值要有后缀f或者F。没有后缀默认为double类型。

double类型也可以添加d或者D后缀。


用于表示溢出或者出错情况的三个特殊浮点值:

\begin{itemize}
        \item   正无穷大  
        \item   负无穷大
        \item   NaN (不是一个数字)
\end{itemize}

例如: 一个正整数除以 0 的结果为正无穷大; 计算 0/0或者负数的平方根结果为NaN。

对应常量为:
\begin{lstlisting}[language=java]
        // float
        Float.POSITIVE_INFINITY;
        Float.NEGATIVE_INFINITY;
        Float.NaN;
        // double
        Double.POSITIVE_INFINITY;
        Double.NEGATIVE_INFINITY;
        Double.NaN;
\end{lstlisting}


所有“非数值”的值都认为是不相同的。

\begin{lstlisting}[language=java]
        if(Double.isNaN(x))   // check whether x is "not a number"
\end{lstlisting}




\textbf{char类型}



































