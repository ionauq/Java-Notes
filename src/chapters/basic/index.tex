\chapter{基本概念}
\label{chap:base}

\section{基础知识}

官方文档   

\href{https://docs.oracle.com/en/java/}{Java Documentation}
\newline

\subsection{常用术语}

JDK:  \menlo{Java Development Kit}  \par
~ 编写java程序的程序员使用的软件

JRE:  \menlo{Java Runtime Environment}  \par
~ 运行java程序的用户使用的软件

Server JRE:  \par
~ 在服务器上运行java程序的软件

SE:  \menlo{Standard Edition}  \par
~ 用于桌面或者简单服务器应用的java平台

EE:  \menlo{Enterprise Edition}  \par
~ 用于复杂服务器应用的java平台 

ME: \menlo{Micro Edition}  \par
~ 用于手机和其他小型设备的java平台  

Java FX:  \par
~ 用于图形化用户界面的一个替代工具包在Oracle的Java SE发布版本中提供

OpenJDK:  \par
~ Java SE的一个免费开源实现,不包含浏览器集成或者JavaFX 

JCP:  \menlo{Java Community Process}  \par
~ Java社区进程

JSR:  \menlo{Java Specification Requests}  \par
~ Java规范请求 由JCP成员向委员会提交的Java发展议案,经过一系列流程后,如果通过最终会体现在未来的Java中

TCK:  \menlo{Technology Compatibility Kit}  \par
~ 技术兼容性测试

\subsection{Java版本}

Java SE是相对于Java EE 和 Java ME,它是Java的标准版。

Java SE8 对应“内部”版本号是1.8.0。\newline

版本号命名规则示例:

\menlo{Java SE 8u31}

~ Java SE 8的第31次更新,内部版本号为1.8.0\_31

\subsection{JDK安装}

Mac可以通过 dmg 文件包安装

安装后相关路径

\begin{lstlisting}[style=cshell]

/Library/Java/JavaVirtualMachines/jdk-*.jdk/Contents/Home/

/usr/libexec/java_home

/Library/Java/JavaVirtualMachines/jdk-11.0.2.jdk/Contents/Home/

# 执行目录
/usr/bin/java

# 库源文件为压缩文件src.zip
/Library/Java/JavaVirtualMachines/jdk-*.jdk/Contents/Home/lib/src.zip

\end{lstlisting}


src.zip 文件包含了所有的公共类的源代码。


\subsection{Java Tools}


* \textbf{javac}

~ You can use the javac tool and its options to read Java class and interface definitions and compile them into bytecode and class files.

~ 您可以使用javac工具及其选项来读取Java类和接口定义,并将它们编译成字节码和类文件。

* \textbf{javap}

~ You use the javap command to disassemble one or more class files.

~ 您可以使用javap命令来反汇编一个或多个类文件。

\hypersetup{linkcolor=blue}

~ 具体使用可参照 \textcolor[rgb]{0,0,1}{\ref{chap:tools_javap}} javap章节

* \textbf{javadoc}

~ You use the javadoc tool and its options to generate HTML pages of API documentation from Java source files.

~ 您可以使用javadoc工具及其选项从Java源文件生成API文档的HTML页面。

* \textbf{java} 

~ You can use the java command to launch a Java application.

~ 您可以使用java命令启动java应用程序。

~ 具体使用可参照 \textcolor[rgb]{0,0,1}{\ref{chap:tools_java}} java章节

* \textbf{appletviewer}

~ You use the appletviewer command to launch the AppletViewer and run applets outside of a web browser.

~ 您可以使用appletviewer命令启动appletviewer并在web浏览器之外运行applet。

* \textbf{jar}

~ You can use the jar command to create an archive for classes and resources, and to manipulate or restore individual classes or resources from an archive.

~ 您可以使用jar命令为类和资源创建存档,并从存档中操作或还原单个类或资源。

* \textbf{jlink}

~ You can use the jlink tool to assemble and optimize a set of modules and their dependencies into a custom runtime image.

~ 您可以使用jlink工具将一组模块及其依赖项组装并优化到自定义运行时映像中。

* \textbf{jmod}

~ You use the jmod tool to create JMOD files and list the content of existing JMOD files.

~ 使用jmod工具创建JMOD文件并列出现有JMOD文件的内容。

* \textbf{jdeps} 

~ You use the jdeps command to launch the Java class dependency analyzer.

~ 使用jdeps命令启动Java类依赖关系分析器。

* \textbf{jdeprscan} 

~ You use the jdeprscan tool as a static analysis tool that scans a jar file (or some other aggregation of class files) for uses of deprecated API elements.

~ 您可以使用jdeprscan工具作为静态分析工具,它扫描一个jar文件(或一些其他类文件的聚合),以便使用废弃的API元素。

* \textbf{jshell}

~ You use the jshell tool to interactively evaluate declarations, statements, and expressions of the Java programming language in a read-eval-print loop (REPL).

~ 您可以使用jshell工具在read-eval-print循环(REPL)中交互式地计算Java编程语言的声明、语句和表达式。

* \textbf{jconsole} 

~ You use the jconsole command to start a graphical console to monitor and manage Java applications.

~ 您可以使用jconsole命令启动图形化控制台来监视和管理Java应用程序。

* \textbf{jmc} 

~ You use the jmc command and its options to launch Java Mission Control. Java Mission Control is a profiling, monitoring, and diagnostics tools suite.

~ 使用jmc命令及其选项启动Java任务控制。Java任务控制是一个分析、监视和诊断工具套件。        

* \textbf{jps}

~ Experimental You use the jps command to list the instrumented JVMs on the target system.

~ 实验性。您可以使用jps命令列出目标系统上已检测的jvm。

* \textbf{jstat} 

~ Experimental You use the jstat command to monitor JVM statistics. 

~ 实验性。使用jstat命令监视JVM统计数据。此命令是实验性的,不受支持。

* \textbf{jstatd} 

~ Experimental You use the jstatd command to monitor the creation and termination of instrumented Java HotSpot VMs. 

~ 实验性。您可以使用jstatd命令监视仪表化Java HotSpot vm的创建和终止。

* \textbf{jmap}

~ Experimental You use the jmap command to print details of a specified process. 

~ 实验性。您可以使用jmap命令打印指定进程的详细信息。

* \textbf{jstack}

~ Experimental You use the jstack command to print Java stack traces of Java threads for a specified Java process. 

~ 实验性。您可以使用jstack命令为指定的Java进程打印Java线程的Java堆栈跟踪。\newline

\notebox{
    更详细的内容请参考    \par 
    \href{https://docs.oracle.com/en/java/javase/11/tools/tools-and-command-reference.html}{tools-and-command-reference}
}


\subsection{规范}

\textbf{Java区分大小写}

Java采用骆驼命名法,CamelCase。

\textbf{数据类型}

Java包含8种基本类型。\par
\begin{itemize}
        \item   4种整型
        \item   2种浮点类型 
        \item   1种表示Unicode编码的字符单元的字符类型char
        \item   1种表示真值的boolean类型
\end{itemize}


\textbf{整型}

\renewcommand\arraystretch{2}
\begin{tabular}{l|l|l}
    类型        &      存储需求        &     取值范围        \\               \hline
    byte       &       1字节          & -128 $\sim$ 127     \\
    short      &       2字节          & -32768 $\sim$ 32767  \\
    int        &       4字节          &  -2147483648 $\sim$ 2147483647  \\
    long       &       8字节          &  -9223372036854775808 $\sim$ 9223372036854775807  \\
\end{tabular}\newline

\notebox{Java 没有任何无符号(unsigned)形式的整型。}

长整型值有一个后缀l或者L。

\begin{lstlisting}[style=cjava]
        long number = 123L;             // number值为:123
\end{lstlisting}

十六进制带前缀0x或者0X

\begin{lstlisting}[style=cjava]
        int number = 0x123;       
\end{lstlisting}

八进制带前缀0

从Java7:

* 加上前缀0b或者0B就可以写二进制数。

\begin{lstlisting}[style=cjava]
        int number = 0b100;             // number值为:4
\end{lstlisting}

* 为数字加下划线,Java编译器会去除这些下划线

\begin{lstlisting}[language=java]
        Sytem.out.println(0_1_0);       // 010表示八进制,输出8
\end{lstlisting}


\cautionbox{
注意JS INT类型的值范围:
\begin{itemize}
        \item   \href{http://speakingjs.com/es5/ch11.html}{Integers in JavaScript} 
        \item   \href{http://speakingjs.com/es5/ch11.html\#safe\_integers}{Safe Integers}
        \end{itemize}
}


\textbf{浮点数}


\renewcommand\arraystretch{2}
\begin{tabular}{l|l|l}
    类型         &      存储需求        &     取值范围        \\               \hline
    float        &       4字节          & 有效位数为6-7位      \\
    double      &       8字节          & 有效位数为15位       \\
\end{tabular}\newline


float类型的值要有后缀f或者F。没有后缀默认为double类型。

double类型也可以添加d或者D后缀。


用于表示溢出或者出错情况的三个特殊浮点值:

\begin{itemize}
        \item   正无穷大  
        \item   负无穷大
        \item   NaN (不是一个数字)
\end{itemize}

例如: 一个正整数除以 0 的结果为正无穷大; 计算 0/0或者负数的平方根结果为NaN。

对应常量为:
\begin{lstlisting}[language=java]
        // float
        Float.POSITIVE_INFINITY;
        Float.NEGATIVE_INFINITY;
        Float.NaN;
        // double
        Double.POSITIVE_INFINITY;
        Double.NEGATIVE_INFINITY;
        Double.NaN;
\end{lstlisting}


所有“非数值”的值都认为是不相同的。

\begin{lstlisting}[language=java]
        if(Double.isNaN(x))   // check whether x is "not a number"
\end{lstlisting}










\textbf{char类型}



常量

关键字final

类常量关键字 static final

ava类中各成员初始化顺序


基类静态代码块、基类静态成员字段并列优先级,按照代码中出现先后顺序执行(只有第一次加载类时执行)

派生类静态代码块、派生类静态成员字段并列优先级,按照代码中出现先后顺序执行(只有第一次加载类时执行)

基类普通代码块、基类普通成员字段并列优先级,按照代码中出现先后顺序执行

基类构造函数

派生类普通代码块、派生类普通成员字段并列优先级,按照代码中出现顺序执行

派生类构造函数

































