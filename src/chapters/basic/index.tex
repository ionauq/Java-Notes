\chapter{基本概念}
\label{chap:base}

\section{基础知识}

官方文档   

\href{https://docs.oracle.com/en/java/}{Java Documentation}
\newline

\subsection{常用术语}

JDK:  \menlo{Java Development Kit}  \par
~ 编写java程序的程序员使用的软件

JRE:  \menlo{Java Runtime Environment}  \par
~ 运行java程序的用户使用的软件

Server JRE:  \par
~ 在服务器上运行java程序的软件

SE:  \menlo{Standard Edition}  \par
~ 用于桌面或者简单服务器应用的java平台

EE:  \menlo{Enterprise Edition}  \par
~ 用于复杂服务器应用的java平台 

ME: \menlo{Micro Edition}  \par
~ 用于手机和其他小型设备的java平台  

Java FX:  \par
~ 用于图形化用户界面的一个替代工具包在Oracle的Java SE发布版本中提供

OpenJDK:  \par
~ Java SE的一个免费开源实现,不包含浏览器集成或者JavaFX 

JCP:  \menlo{Java Community Process}  \par
~ Java社区进程

JSR:  \menlo{Java Specification Requests}  \par
~ Java规范请求 由JCP成员向委员会提交的Java发展议案,经过一系列流程后,如果通过最终会体现在未来的Java中

TCK:  \menlo{Technology Compatibility Kit}  \par
~ 技术兼容性测试

\subsection{Java版本}

Java SE是相对于Java EE 和 Java ME,它是Java的标准版。

Java SE8 对应“内部”版本号是1.8.0。\newline

版本号命名规则示例:

\menlo{Java SE 8u31}

~ Java SE 8的第31次更新,内部版本号为1.8.0\_31

\subsection{JDK安装}

Mac可以通过 dmg 文件包安装

安装后相关路径

\begin{lstlisting}[style=cshell]

/Library/Java/JavaVirtualMachines/jdk-*.jdk/Contents/Home/

/usr/libexec/java_home

/Library/Java/JavaVirtualMachines/jdk-11.0.2.jdk/Contents/Home/

# 执行目录
/usr/bin/java

# 库源文件为压缩文件src.zip
/Library/Java/JavaVirtualMachines/jdk-*.jdk/Contents/Home/lib/src.zip

\end{lstlisting}

src.zip 文件包含了所有的公共类的源代码。

\subsubsection{设置环境变量}

设置JAVA\_HOME

\begin{lstlisting}[style=cshell]
        export JAVA_HOME="$(/usr/libexec/java_home -v 1.8.0_201)"       # 以版本1.8.0_201为例
        export PATH=$JAVA_HOME/bin:$PATH
\end{lstlisting}


\section{Java Tools}

\subsubsection{Create and Build Applications}

* \textbf{javac}

~ You can use the javac tool and its options to read Java class and interface definitions and compile them into bytecode and class files.

~ 您可以使用javac工具及其选项来读取Java类和接口定义,并将它们编译成字节码和类文件。

\hypersetup{linkcolor=blue}

~ 具体使用可参照 \textcolor{blue}{\ref{chap:tools_javac}} javac章节

* \textbf{javap}

~ You use the javap command to disassemble one or more class files.

~ 您可以使用javap命令来反汇编一个或多个类文件。

\hypersetup{linkcolor=blue}

~ 具体使用可参照 \textcolor[rgb]{0,0,1}{\ref{chap:tools_javap}} javap章节

* \textbf{java} 

~ You can use the java command to launch a Java application.

~ 您可以使用java命令启动java应用程序。

~ 具体使用可参照 \textcolor[rgb]{0,0,1}{\ref{chap:tools_java}} java章节

* \textbf{jar}

~ You can use the jar command to create an archive for classes and resources, and to manipulate or restore individual classes or resources from an archive.

~ 您可以使用jar命令为类和资源创建存档,并从存档中操作或还原单个类或资源。

* \textbf{jdeps} 

~ You use the jdeps command to launch the Java class dependency analyzer.

~ 使用jdeps命令启动Java类依赖关系分析器。

* \textbf{javadoc}

~ You use the javadoc tool and its options to generate HTML pages of API documentation from Java source files.

~ 您可以使用javadoc工具及其选项从Java源文件生成API文档的HTML页面。

\subsubsection{Monitor Java Applications}

* \textbf{jconsole} 

~ You use the jconsole command to start a graphical console to monitor and manage Java applications.

~ 您可以使用jconsole命令启动图形化控制台来监视和管理Java应用程序。

* \textbf{jvisualvm}


\subsubsection{Monitor the JVM}

* \textbf{jps}

~ Experimental You use the jps command to list the instrumented JVMs on the target system.

~ 实验性。您可以使用jps命令列出目标系统上已检测的jvm。

~ 具体使用可参照 \textcolor[rgb]{0,0,1}{\ref{chap:tools_jps}} jps章节

* \textbf{jstat} 

~ Experimental You use the jstat command to monitor JVM statistics. 

~ 实验性。使用jstat命令监视JVM统计数据。此命令是实验性的,不受支持。

* \textbf{jstatd} 

~ Experimental You use the jstatd command to monitor the creation and termination of instrumented Java HotSpot VMs. 

~ 实验性。您可以使用jstatd命令监视仪表化Java HotSpot vm的创建和终止。

* \textbf{jmc} 

~ You use the jmc command and its options to launch Java Mission Control. Java Mission Control is a profiling, monitoring, and diagnostics tools suite.

~ 使用jmc命令及其选项启动Java任务控制。Java任务控制是一个分析、监视和诊断工具套件。        

\subsubsection{Troubleshooting Tools}

* \textbf{jcmd}

~ You use the jcmd utility to send diagnostic command requests to a running Java Virtual Machine (JVM).

* \textbf{jdb}

~ You use the jdb command and its options to find and fix bugs in Java platform programs.

* \textbf{jinfo}

~ Experimental You use the jinfo command to generate Java configuration information for a specified Java process. 

* \textbf{jmap}

~ Experimental You use the jmap command to print details of a specified process. 

~ 实验性。您可以使用jmap命令打印指定进程的详细信息。

* \textbf{jstack}

~ Experimental You use the jstack command to print Java stack traces of Java threads for a specified Java process. 

~ 实验性。您可以使用jstack命令为指定的Java进程打印Java线程的Java堆栈跟踪。\newline

\subsubsection{Others}

* \textbf{jlink}

~ You can use the jlink tool to assemble and optimize a set of modules and their dependencies into a custom runtime image.

~ 您可以使用jlink工具将一组模块及其依赖项组装并优化到自定义运行时映像中。

* \textbf{jmod}

~ You use the jmod tool to create JMOD files and list the content of existing JMOD files.

~ 使用jmod工具创建JMOD文件并列出现有JMOD文件的内容。

* \textbf{jdeprscan} 

~ You use the jdeprscan tool as a static analysis tool that scans a jar file (or some other aggregation of class files) for uses of deprecated API elements.

~ 您可以使用jdeprscan工具作为静态分析工具,它扫描一个jar文件(或一些其他类文件的聚合),以便使用废弃的API元素。

* \textbf{jshell}

~ You use the jshell tool to interactively evaluate declarations, statements, and expressions of the Java programming language in a read-eval-print loop (REPL).

~ 您可以使用jshell工具在read-eval-print循环(REPL)中交互式地计算Java编程语言的声明、语句和表达式。


\notebox{
    更详细的内容请参考    \par 
    \href{https://docs.oracle.com/en/java/javase/11/tools/tools-and-command-reference.html}{tools-and-command-reference}
}


