\chapter{字符串}
\label{chap:string}

    Strings are constant; their values cannot be changed after they are created.

\section{字符串对象}

String对象是不可改变的,具有恒定性。

在Java中String对象可以认为是char数组的延伸和进一步的封装。

\begin{noteblock}
    这里所说的char数组不是C语言意义上的字符型数组,而是大致类似于C语言中的 char* 指针。 \par
    \begin{lstlisting}[style=cjava]
    String str = "Java学习笔记"; 
    \end{lstlisting}
    \begin{lstlisting}[style=cjava]
    char* str = "Java学习笔记";
    \end{lstlisting}
\end{noteblock}

\begin{lstlisting}[style=cjava]

    public final class String implements java.io.Serializable, Comparable<String>, CharSequence {

    /** The value is used for character storage. */
    private final char value[];

    /*
    * @param  value             Array that is the source of characters
    * @param  offset            The initial offset
    * @param  count             The length
    *
    * @throws  IndexOutOfBoundsException
    *          If the {@code offset} and {@code count} arguments index
    *          characters outside the bounds of the {@code value} array
    */
    public String(char value[], int offset, int count) {
        if (offset < 0) {
            throw new StringIndexOutOfBoundsException(offset);
        }
        if (count <= 0) {
            if (count < 0) {
                throw new StringIndexOutOfBoundsException(count);
            }
            if (offset <= value.length) {
                this.value = "".value;
                return;
            }
        }
        // Note: offset or count might be near -1>>>1.
        if (offset > value.length - count) {
            throw new StringIndexOutOfBoundsException(offset + count);
        }
        this.value = Arrays.copyOfRange(value, offset, offset+count);
    }

    }

\end{lstlisting}

通过上面String类的实现代码可以发现String类和value数组都是final类型,这就保证了String对象的不变性。这种不变性可以带来极大的好处。

\begin{itemize}
    \item 保证对String对象的任意操作都不会改变原字符串
    \item 意味着操作字符串不会出现线程同步问题
    \item 成就了字符串驻留以及共享(常量池)
\end{itemize}











































