\chapter{字符串}
\label{chap:string}

    Strings are constant; their values cannot be changed after they are created.

\section{字符串对象}

String对象不可改变,具有恒定性。

每个String对象都有常量值。

字符串字面常量是对String类对象的引用。

在Java中String对象可以认为是char数组的延伸和进一步的封装。

\begin{noteblock}
    这里所说的char数组不是C语言意义上的字符型数组,而是大致类似于C语言中的 char* 指针。 \par
    \begin{lstlisting}[style=cjava]
    String str = "Java学习笔记"; 
    \end{lstlisting}
    \begin{lstlisting}[style=cjava]
    char* str = "Java学习笔记";
    \end{lstlisting}
\end{noteblock}

\begin{lstlisting}[style=cjava]

    public final class String implements java.io.Serializable, Comparable<String>, CharSequence {

    /** The value is used for character storage. */
    private final char value[];

    /*
    * @param  value             Array that is the source of characters
    * @param  offset            The initial offset
    * @param  count             The length
    *
    * @throws  IndexOutOfBoundsException
    *          If the {@code offset} and {@code count} arguments index
    *          characters outside the bounds of the {@code value} array
    */
    public String(char value[], int offset, int count) {
        if (offset < 0) {
            throw new StringIndexOutOfBoundsException(offset);
        }
        if (count <= 0) {
            if (count < 0) {
                throw new StringIndexOutOfBoundsException(count);
            }
            if (offset <= value.length) {
                this.value = "".value;
                return;
            }
        }
        // Note: offset or count might be near -1>>>1.
        if (offset > value.length - count) {
            throw new StringIndexOutOfBoundsException(offset + count);
        }
        this.value = Arrays.copyOfRange(value, offset, offset+count);
    }

    }

\end{lstlisting}

通过上面String类的实现代码可以发现String类和value数组都是final类型,这就保证了String对象的不变性。这种不变性可以带来极大的好处。

\begin{itemize}
    \item 保证对String对象的任意操作都不会改变原字符串
    \item 意味着操作字符串不会出现线程同步问题
    \item 成就了字符串驻留以及共享(常量池)
\end{itemize}



\section{字符串连接操作符 + }


如果字符串连接操作符运算的结果不是编译时的常量表达式,那么该操作符会隐式地创建新的String对象。


如果只有一个操作数表达式是String 类型,那么就会在另一个操作数上执行字符串转换以在运行时产生字符串。 
对于简单类型Java还可以通过直接将简单类型转换为字符串而优化掉包装器对象的创建。

\textcolor{codepurple}{+}操作字符在语法上是左结合。

例如:

\begin{lstlisting}[style=cjava]

    String first = 1 + 2 + "Java";
    String second = "Java" + 1 + 2;
    String three = "Java" + null;
    System.out.println(first);
    System.out.println(second);
    System.out.println(three);

    /** 
    * Output: 
    * 
    * 3Java
    * Java12
    * Javanull
    * 
    */

\end{lstlisting}


在字符串转换方面要特别注意引用类型转行。

如果该引用是null,那么它转换为字符串"null"。

如果其他引用类型,则调用其对象上的toString()方法;如果调用toString()方法的结果是null,那么就用字符串"null"代替。


\section{字符串常量池}

\href{https://tech.meituan.com/2014/03/06/in-depth-understanding-string-intern.html}{深入解析String\#intern}


\section{字符串拼接函数}


\section{特殊注意事项}

\subsection{String.format 和 MessageFormat.format区别}


\subsection{MessageFormat.format中传入long类型时问题}


\subsection{String.split()使用单字符传入特殊符号时问题}

\begin{lstlisting}[style=cjava]

    /**
    * Splits this string around matches of the given regular expression.
    */
    public String[] split(String regex);

\end{lstlisting}

regex 参数是正则表达式。如果使用单字符时并出现如下字符时一定要做转义,使用 \menlo{$\backslash$$\backslash$} 转义。

\begin{lstlisting}[style=cjava]
    . $ | ( ) [ { ^ ? * + \\
\end{lstlisting}


\begin{lstlisting}[style=cjava]

    String origin = "a.b.c.d";
    //错误使用
    String[] errorArray = origin.split(".");
    for (int i = 0; i < errorArray.length; i++) {
        System.out.println(i);
    }
    //正确使用;使用\\转义
    String[] rightArray = origin.split("\\.");
    for (String s : rightArray) {
        System.out.println(s);
    }

\end{lstlisting}





























































