\section{uptime}
\label{chap:linux_uptime}

uptime

当前时间系统运行了多长时间、多少用户登陆;以及过去1分钟、5分钟、15分钟的系统平均负载。

\begin{lstlisting}[language=cshell]
    $ uptime

    # 20:08:43 up 11 days,  6:06,  0 users,  load average: 21.76, 21.24, 19.93

    # 20:08:43      系统当前时间
    # up 11 days,  6:06     系统运行时间
    # 0 users       当前登陆用户数
    # load average: 21.76, 21.24, 19.93     过去1分钟、5分钟、15分钟的系统平均负载
\end{lstlisting}


系统负载

系统负载是处于可运行(runnable)或不可中断(uninterruptable)状态的进程的平均数。\par
可运行(runnable)状态的进程要么正在使用CPU,要么在等待使用CPU。不可中断状态的进程则正在等待某些I/O访问,例如等待磁盘IO。 \par
负载均值的意义根据系统中 CPU 的数量不同而不同,负载为 1 对于一个只有单 CPU 的系统来说意味着负载满了,而对于一个拥有 4 CPU 的系统来说则意味着 75\% 的时间里都是空闲的。


\par
OPTIONS

-p, --pretty \par
\qquad 以漂亮的格式显示正常运行时间

\begin{lstlisting}[language=cshell]
    uptime -p

    # up 17 weeks, 1 day, 11 hours, 37 minutes
\end{lstlisting}


-s, --since \par
\qquad 以yyyy-mm-dd HH:MM:SS格式显示系统启动时间

\begin{lstlisting}[language=cshell]
    uptime -s

    # 2019-11-18 23:30:43
\end{lstlisting}












