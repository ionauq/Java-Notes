函数式接口

关于函数式接口

如果一个接口只有一个抽象方法,那么该接口就是一个函数式接口

如果我们在某个接口上声明了FunctionalInterface注解,那么编译器就会按照函数式接口的定义来要求该接口

如果某个接口只有一个抽象方法,但我们并没有给该接口声明FunctionalInterface注解,那么编译器依旧会将该接口看作是函数式接口

\begin{lstlisting}[style=cjava]
    List<Integer> list = Arrays.asList(1,2,3,4,5);

    list.forEach(new Consumer<Integer>(){
        @Override
        public void accept(Integer integer){
            System.out.println(integer);
        }
    });
\end{lstlisting}

在java中Lambda表达式是对象,他们必须依附与一类特别的对象类型-函数式接口(functional interface)

That instances of functional interfaces can be created with lambda expressions, method references, or constructor references.

函数式接口的实例可以通过lambda表达式,方法引用和构造函数引用创建。


外部迭代和内部迭代

Java lambda表达式是一种匿名函数;它是没有声明的方法,即没有访问修饰符,返回值声明和名字。

lambda操作符: ->
lambda左边:接口中抽象方法的形参列表
lamdba右边:重写抽象方法的方法体


当只有一个参数,并且类型可推导时,圆括号()可省略。例如: a-> return a*a

lambda 表达式的主体可包含零条或多条语句

如果lambda表达式的主体只有一条语句,花括号{} 可省略。匿名函数的返回类型与该主体表达式一致

如果Lambda表达式的主体包含一条以上语句,则表达式必须包含在花括号{}中。匿名函数的返回类型与代码块的返回类型一致,若没有返回值则为空



Function<T, R>
 
BiFunction<T, U, R>


