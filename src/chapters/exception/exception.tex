\chapter{异常处理}

\section{异常}

\label{chap:exception}

异常的抛出同其他对象的创建一样,将使用new在堆上创建异常对象。然后当前的执行路径(它不能继续执行下去)被终止,并且从当前环境中弹出对异常对象的引用。
此时异常处理机制接管程序,并开始寻找一个恰当的地方来继续执行程序。而这个恰当的地方就是“异常处理程序”。它的任务是将程序从错误状态中恢复,以使程序能
要么换一种方式运行,要么继续运行下去。

Throwable类是所有异常的基类。

//此处有类图

Throwable Error Exception RuntimeException 非RuntimeException




\section{异常说明}

异常说明使用附加关键字throws。



重抛异常会把异常抛给上一级环境中的异常处理程序。同一个try块的后续catch子句将被忽略。




\begin{lstlisting}[language=java]
    
/**
 * The {@code Throwable} class is the superclass of all errors and
 * exceptions in the Java language. Only objects that are instances of this
 * class (or one of its subclasses) are thrown by the Java Virtual Machine or
 * can be thrown by the Java {@code throw} statement. Similarly, only
 * this class or one of its subclasses can be the argument type in a
 * {@code catch} clause.
 *
 * For the purposes of compile-time checking of exceptions, {@code
 * Throwable} and any subclass of {@code Throwable} that is not also a
 * subclass of either {@link RuntimeException} or {@link Error} are
 * regarded as checked exceptions.
 *
 * <p>Instances of two subclasses, {@link java.lang.Error} and
 * {@link java.lang.Exception}, are conventionally used to indicate
 * that exceptional situations have occurred. Typically, these instances
 * are freshly created in the context of the exceptional situation so
 * as to include relevant information (such as stack trace data).
 *
 * <p>A throwable contains a snapshot of the execution stack of its
 * thread at the time it was created. It can also contain a message
 * string that gives more information about the error. Over time, a
 * throwable can {@linkplain Throwable#addSuppressed suppress} other
 * throwables from being propagated.  Finally, the throwable can also
 * contain a <i>cause</i>: another throwable that caused this
 * throwable to be constructed.  The recording of this causal information
 * is referred to as the <i>chained exception</i> facility, as the
 * cause can, itself, have a cause, and so on, leading to a "chain" of
 * exceptions, each caused by another.
 *
 * <p>One reason that a throwable may have a cause is that the class that
 * throws it is built atop a lower layered abstraction, and an operation on
 * the upper layer fails due to a failure in the lower layer.  It would be bad
 * design to let the throwable thrown by the lower layer propagate outward, as
 * it is generally unrelated to the abstraction provided by the upper layer.
 * Further, doing so would tie the API of the upper layer to the details of
 * its implementation, assuming the lower layer's exception was a checked
 * exception.  Throwing a "wrapped exception" (i.e., an exception containing a
 * cause) allows the upper layer to communicate the details of the failure to
 * its caller without incurring either of these shortcomings.  It preserves
 * the flexibility to change the implementation of the upper layer without
 * changing its API (in particular, the set of exceptions thrown by its
 * methods).
 *
 * <p>A second reason that a throwable may have a cause is that the method
 * that throws it must conform to a general-purpose interface that does not
 * permit the method to throw the cause directly.  For example, suppose
 * a persistent collection conforms to the {@link java.util.Collection
 * Collection} interface, and that its persistence is implemented atop
 * {@code java.io}.  Suppose the internals of the {@code add} method
 * can throw an {@link java.io.IOException IOException}.  The implementation
 * can communicate the details of the {@code IOException} to its caller
 * while conforming to the {@code Collection} interface by wrapping the
 * {@code IOException} in an appropriate unchecked exception.  (The
 * specification for the persistent collection should indicate that it is
 * capable of throwing such exceptions.)
 *
 * <p>A cause can be associated with a throwable in two ways: via a
 * constructor that takes the cause as an argument, or via the
 * {@link #initCause(Throwable)} method.  New throwable classes that
 * wish to allow causes to be associated with them should provide constructors
 * that take a cause and delegate (perhaps indirectly) to one of the
 * {@code Throwable} constructors that takes a cause.
 *
 * Because the {@code initCause} method is public, it allows a cause to be
 * associated with any throwable, even a "legacy throwable" whose
 * implementation predates the addition of the exception chaining mechanism to
 * {@code Throwable}.
 *
 * <p>By convention, class {@code Throwable} and its subclasses have two
 * constructors, one that takes no arguments and one that takes a
 * {@code String} argument that can be used to produce a detail message.
 * Further, those subclasses that might likely have a cause associated with
 * them should have two more constructors, one that takes a
 * {@code Throwable} (the cause), and one that takes a
 * {@code String} (the detail message) and a {@code Throwable} (the
 * cause).
 *
 * @author  unascribed
 * @author  Josh Bloch (Added exception chaining and programmatic access to
 *          stack trace in 1.4.)
 * @jls 11.2 Compile-Time Checking of Exceptions
 * @since JDK1.0
 */

\end{lstlisting}




\section{异常捕获}

try.catch.finally

Java7开始,可以在一个catch表达式中对多种类型异常进行合并捕获。

\begin{lstlisting}[language=java]
    public int getBinaryInt(String number) {
        int result = -1;
        result = result / 2;
        try {
            result = Integer.parseInt(number, 2);
        } catch (NumberFormatException | ArithmeticException e) {
                
        }

        return result;
    }
\end{lstlisting}


