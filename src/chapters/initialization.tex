\chapter{对象}

\label{chap:object}

\section{初始化}


静态初始化只有在对象被创建或者第一次访问静态数据时才会被初始化。

初始化的顺序时先静态对象(如果他们尚未因前面的对象创建过程而被初始化),而后才是“非静态”对象。

以Dog类为示例总结一下对象的创建过程:

1. 当首次创建类型为Dog的对象或者Dog类的静态方法/静态域首次被访问时,java解释器必须查找类路径,以定位Dog.class文件

2.然后载入Dog.class ,有关静态初始化的所有动作都会执行。因此,静态初始化只在Class对象首次加载的时候进行一次。按照顺序。

3. 当用new Dog() 创建对象的时候,首先将在堆上为Dog对象分配足够的存储空间。

4. 这块存储空间会被清零,这就自动将Dog对象中的所有基本类型数据都设置成了默认值(数字为0,boolean为false),而引用则被设置成null(例如String)

5. 执行所有出现于字段定义处的初始化动作

6. 执行构造器(会涉及到继承的问题)


总结


基类静态代码块、基类静态成员字段并列优先级,按照代码中出现先后顺序执行(只有第一次加载类时执行)

派生类静态代码块、派生类静态成员字段并列优先级,按照代码中出现先后顺序执行(只有第一次加载类时执行)

基类普通代码块、基类普通成员字段并列优先级,按照代码中出现先后顺序执行

基类构造函数

派生类普通代码块、派生类普通成员字段并列优先级,按照代码中出现顺序执行

派生类构造函数

\subsection{对象拷贝}

对象的拷贝分为 shallow copy 和 deep copy 。

shallow copy 既浅拷贝。




https://zhuanlan.zhihu.com/p/26964202

\href{https://zgxxx.github.io/2019/02/27/20190227/}{clone}

\href{https://howtodoinjava.com/java/cloning/a-guide-to-object-cloning-in-java/}{clone}

PHP与Java一样,对象的对象属性都只是引用拷贝.

php的拷贝是通过clone关键字实现。





