\section{synchronized}
\label{chap:synchronized}

JDK为什么要设计synchronized?

多线程编程中,有可能出现多个线程同时访问同一个共享的,可变资源的情况。这种资源可能是对象、变量、文件等。

\textbf{共享}:  资源可以由多个线程同时访问

\textbf{可变}: 资源可以在起生命周期内被修改

引出的问题:

由于线程执行的过程是不可控的,所以需要采用同步机制来协同对对象可变状态的访问


加锁的目的:

序列号访问临界资源,即同一时刻只能有一个线程访问临界资源(同步互斥访问)


\notebox{
    synchronized 属于隐式锁 \par
    隐式锁:不需要自己通过写代码取加锁跟解锁
}

synchronized使用与实现

使用方式

\begin{itemize}
    \item 同步实例方法,锁是当前实例对象
    \item 同步类方法,锁是当前类对象
    \item 同步代码块, 锁是括号里面的对象
\end{itemize}


实现方式

synchronized 是JVM内置锁,通过内部对象Monitor(监视器锁)实现,基于进入与退出Monitor对象实现方法与代码块同步。监视器锁的实现依赖底层操作系统的Mutex lock(互斥锁)实现。

JVM对象加锁原理

认识对象的内存结构

oop.hpp

\href{https://gist.github.com/arturmkrtchyan/43d6135e8a15798cc46c}{Object Header}


#TODO:


> 实例对象内存存储是怎样的?
> 对象的实例存在堆区,对象元数据存在元空间(方法区),对象的引用存在与栈空间


锁的种类


JDK1.6之前是重量级锁,之后对synchronized对实现进行了优化,如适应性自旋锁,轻量级锁和偏向锁。默认开启偏向锁。

开启偏向锁: -XX:+UseBiasedLocking -XX:BiasedLockingStartupDelay=0

锁的升级过程


无锁 -> 偏向锁 -> 轻量级锁  -> 重量级锁


为什么不直接上重量级锁

上重量级锁会涉及到线程上下文切换?线程上下文切换会涉及到操作系统用户态到内核态的切换。



JVM 原子操作

lock,read,load,use,assign,store,write,unlock


read: 读取,从主内存读取数据

load: 载入,将主内存读取到的数据写入工作内存

use: 使用 从工作内存读取数据来计算

assign: 赋值 将计算好的值重新赋值到工作内存中

store: 存储 将工作内存数据写入主内存

write: 写入 将store过去的变量值赋值给主内存中的变量

lock: `锁定` 将主内存变量加锁,标识为线程独占状态

unlock: `解锁` 将主内存变量解锁,解锁后其他线程可以锁定该变量




MESI缓存一致协议








https://www.slideshare.net/cnbailey/memory-efficient-java

