\chapter{内部类}
\label{chap:inner_class}

可以将一个类的定义放在另一个类的定义内部,这就是内部类。

内部类的对象与制造它的外围对象(enclosing object)之间就有一种联系,它能访问其外围对象的所有成员,而不要任何特殊条件。
内部类还拥有其外围类的所有元素的访问权。


\section{.this和.new}




Anonymous Inner Class

It is an inner class without a name and for which only a single object is created.
An anonymous inner class can be useful when making an instance of an object with certain “extras” such as overloading methods of a class or interface, without having to actually subclass a class.


are mainly created in two ways:

. Class (abstract or concrete)
. interface

The syntax of an anonymous class expression is like the invocation of a constructor, except that there is a class definition contained in a block of code.
匿名类表达式的语法类似于调用构造函数,只是在代码块中包含了一个类定义。

匿名内部类的创建格式

\begin{lstlisting}[style=cjava]
    new class(arguments) | interface()
    {
        //匿名内部类的类体部分
    }
\end{lstlisting}

okHttp中匿名类示例

\begin{lstlisting}[style=cjava]
    public abstract class Internal {
        public abstract void addLenient(Headers.Builder builder, String line);
    }

    Internal.instance = new Internal() {
        @Override public void addLenient(Headers.Builder builder, String line) {
        builder.addLenient(line);
        }
    }
\end{lstlisting}


Difference between Normal/Regular class and Anonymous Inner class:

A normal class can implement any number of interfaces but anonymous inner class can implement only one interface at a time.
A regular class can extend a class and implement any number of interface simultaneously. But anonymous Inner class can extend a class or can implement an interface but not both at a time.
For regular/normal class, we can write any number of constructors but we cant write any constructor for anonymous Inner class because anonymous class does not have any name and while defining constructor class name and constructor name must be same.


\subsection{嵌套类}

将内部类声明为static,内部类对象与其外围类对象之间则没有联系。

1. 要创建嵌套类的对象,并不需要其外围类的对象。
2. 不能从嵌套类的对象中访问非静态的外围类对象。



