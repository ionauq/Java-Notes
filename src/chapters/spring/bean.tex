\chapter{Spring}

\subsection{使用应用上下文}

AnnotationConfigApplicationContext:从一个或多个基于Java的配置类中加载Spring应用上下文。

AnnotationConfigWebApplicationContext:从一个或多个基于Java的配置类中加载Spring Web应用上下文。

ClassPathXmlApplicationContext:从类路径下的一个或多个XML配置文件中加载上下文定义,把应用上下文的定义文件做为类资源。

FileSystemXmlApplicationContext:从文件系统下的一个或者多个XML配置文件中加载上下文定义。

XmlWebApplicationContext:从web应用下的一个或者多个Xml配置文件中加载上下文定义。


自动装配方法

@Component 表明该类会做为组件类,并告知Spring要为这个类创建Bean。
@ComponentScan 启用组件扫描,查找带有@Component注解的类。
或者使用xml格式。<context:component-scan base-package=""/>

@AutoWired 自动装配Bean。可以用在属性、构造器、属性的Setter方法上。其实它可以用在任何方法上。为了避免
没有匹配的bean而抛出异常可以使用@AutoWired(required=false),然而这样需要添加非null检查。

显失装配

Java和XML方式


\subsection{处理自动装配的歧义性}
如果多个类实现了同一接口,而在使用接口做为属性等时,spring不知道加载哪一个类。

使用限定符@Qualifier在注入时指定注入进入哪个bean。


\subsection{bean的作用域}

在默认情况下,spring应用上下文中所有bean都是做为单例(singleton)的形式创建的。

Spring定义了多种作用域

\begin{itemize}
    \item 单例(Singleton)   在整个应用中,只创建bean的一个实例
    \item 原型(Prototype)   每次注入或者通过Spring应用上下文获取的时候,都会创建一个新的bean实例
    \item 会话(Session)     在Web应用中,为每一个会话创建一个bean实例
    \item 请求(Request)     在Web应用中,为每个请求创建一个bean实例
\end{itemize}

使用@Scope注解修改原型作用域。






