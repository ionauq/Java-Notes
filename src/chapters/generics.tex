\chapter{范型}

\label{chap:generics}

\section{基本概念}

范型实现了参数化类型的概念。

\section{范型类}


\section{范型方法}

\section{通配符类型}

? extends T 

? super T

无限定通配符

<?>



\section{范型代码和虚拟机}

虚拟机没有泛型类型对象——所有对象都属于普通类。

\subsection{类型擦除}
 
擦除类型变量并替换为限定类型(无限定的变量用Object)。

原始类型用第一个限定的类型变量来替换,如果没有给定限定就用Object替换。

例如类Pair<T>中的类型变量没有显式的限定,因此原始类型用Object替换T。为了提高效率应该将标签(tagging)接口(即没有方法的接口)
放在边界列表的末尾。

\begin{lstlisting}[style=cjava,label=useless]
    public class Interval<T extends Comparable & Serializable> implements Serializable
    {
        private T lower;
        private T upper;

        public Interval(T first, T second){

        }
    }

    public class Interval implements Serializable
    {
        private Comparable lower;
        private Comparable upper;

        public Interval(Comparable first, Comparable second){

        }
    }
\end{lstlisting}

\subsection{翻译泛型表达式}

当程序调用泛型方法时,如果擦除返回类型,编译器插入强制类型转换。

编译器会把方法调用翻译为两条虚拟机指令:

1. 对原始方法的调用
2. 将返回的Object类型(无限定类型为例)强制转化为特定类型。


桥方法被合成来保持多态。


\subsection{约束和局限性}

\subsubsection{不能使用基本类型实例化类型参数}

\subsubsection{运行时类型查询只适用于原始类型}

\subsubsection{不能创建参数化类型的数组}

可以声明通配类型的数组,然后进行类型转换:

\begin{lstlisting}[style=cjava,label=useless]
    Pair<String>[] table = new Pair<String>[10]; // Error
    Pair<String>[] table = (Pair<String>[]) new Pair<?>[10];
\end{lstlisting}

如果需要收集参数化类型对象,只有一种安全而有效的方法:使用ArrayList:ArrayList<Pair<String>>。


\subsubsection{Varargs警告}
\begin{lstlisting}[style=cjava,label=useless]
    public static <T> void addAll(Collection<T> coll, T... ts){

    }
\end{lstlisting}

\subsubsection{不能实例化类型变量}
 不能使用像new T(...), new T[...]或者T.class这样的表达式中的类型变量。
 错误实例
 \begin{lstlisting}[style=cjava,label=useless]
    public Pair(){
        first = new T();    second = new T();   // Errors
    }
 \end{lstlisting}

 \begin{lstlisting}[style=cjava,label=useless]
    Pair<String> p = Pair.makePair(String::new);

    public static <T> Pair<T> makePair(Supplier<T> constr){
        return new Pair<>(constr.get(), constr.get());
    }

    Pair<String> otherPair = Pair.makePairOther(String.class);
    public static <T> Pair<T> makePairOther(Class<T> cl){
        try{
            return new Pair<>(cl.newInstance(), cl.newInstance());
        }catch(Exception ex){
            return null;
        }
    }
 \end{lstlisting}

 \subsubsection{不能构造泛型数组}
 \subsubsection{泛型类的静态上下文中类型变量无效}

 \subsubsection{不能抛出或者捕获泛型类的实例}

 \subsubsection{可以消除对受查异常的检查}
 





