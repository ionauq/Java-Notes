\section{javap命令}
\label{chap:tools_javap}

use the javap command to disassemble one or more class files.

\begin{lstlisting}[language=cshell]

用法: javap <options> <classes>

  -version                 版本信息
  -v  -verbose             输出附加信息
  -l                       输出行号和本地变量表
  -public                  仅显示公共类和成员
  -protected               显示受保护的/公共类和成员
  -package                 显示程序包/受保护的/公共类和成员 (默认)
  -p  -private             显示所有类和成员
  -c                       对代码进行反汇编
  -s                       输出内部类型签名
  -sysinfo                 显示正在处理的类的系统信息 (路径, 大小, 日期, MD5 散列)
  -constants               显示最终常量
  -classpath <path>        指定查找用户类文件的位置
  -cp <path>               指定查找用户类文件的位置
  -bootclasspath <path>    覆盖引导类文件的位置

\end{lstlisting}

示例:

\begin{lstlisting}[language=java]

javap -c Main.class

\end{lstlisting}

\subsection{Java字节码} 

Instruction set 


instructions fall into a number of broad groups:

\begin{itemize}
    \item 加载和存储指令(Load and store)(e.g. aload\_0, istore)
    \item 算术与逻辑指令(Arithmetic and logic) (e.g. ladd, fcmpl)
    \item 类型转换指令(Type conversion) (e.g. i2b, d2i)
    \item 对象创建与操作指令(Object creation and manipulation) (new, putfield)
    \item 堆栈操作指令(Operand stack management) (e.g. swap, dup2)
    \item 控制转移指令(Control transfer) (e.g. ifeq, goto)
    \item 方法调用与返回指令(Method invocation and return) (e.g. invokespecial, areturn)
\end{itemize}

大多数的指令有前缀和(或)后缀来表明其操作数的类型。如下表:

Many instructions have prefixes and/or suffixes referring to the types of operands they operate on.

\begin{table}[H]
    \centering
    \caption{前缀/后缀类型对照表}
    \begin{tabular}{|l|l|}
    \hline
    Prefix/suffix & Operand type    \\ \hline
    i           & integer           \\ \hline
    l           & long             \\ \hline
    s           & short             \\ \hline
    b           & byte              \\ \hline
    c           & character         \\ \hline
    f           & float             \\ \hline
    d           & double            \\ \hline
    a           & reference         \\ \hline
    \end{tabular}
    \end{table}


Java bytecode

\begin{table}[H]
    \centering
    \caption{Java字节码}
    \begin{tabular}{|l|l|l|}
    \hline
    mnemonic    & stack [before]->[after]   & description                                           \\ \hline
    aload\_0    & → objectref               & load a reference onto the stack from local variable 0  \\ \hline
    \end{tabular}
    \end{table}


        
\href{https://en.wikipedia.org/wiki/Java_bytecode_instruction_listings}{bytecode}

\href{http://blog.jamesdbloom.com/JavaCodeToByteCode_PartOne.html}{code}

\href{https://docs.oracle.com/javase/specs/jvms/se8/html/jvms-4.html#jvms-4.10.1.9}{jvms}
  