\section{jps命令}
\label{chap:tools_jps}

Lists the instrumented Java Virtual Machines (JVMs) on the target system. 

只适用于HotSpot虚拟机。

\begin{lstlisting}[language=cshell]

jps [ options ] [ hostid ]

\end{lstlisting}

hostid可以是进程的标识符或者类似于URL的 [protocol:][[//]hostname][:port][/servername] 地址

如果jps命令没有指定hostid参数,那么工具只搜索本机运行的JVM。如果指定了hostid,那么他将通过指定的地址来搜索JVM。
使用指定hostid的主机必须运行着jstatd进程。

\subsection{选项} 

\menlo{-q}

只输出JVM标识符的列表,不输出类名、JAR文件名以及传递给main方法的参数。


\menlo{-m}

输出传递给mian方法的参数。嵌入式JVMS可能输出为空。


\menlo{-l}

输出应用程序主类的完整包名或应用程序JAR文件的完整路径名。

\menlo{-v}

输出传递给JVM的参数。

\menlo{-V}

只输出本地JVM标识符,不输出类名、JAR文件以及传递给main方法的参数。


\menlo{-Joption}

将选项传递给JVM,其中的选项是Java应用程序启动程序参考页面中描述的选项之一。例如,-J-Xms48m将启动内存设置为48 MB。



输出格式:

\begin{lstlisting}[language=cshell]

lvmid [ [ classname | JARfilename | "Unknown"] [ arg* ] [ jvmarg* ] ]

\end{lstlisting}



示例:

\begin{lstlisting}[language=cshell]
jps

    18027 Java2Demo.JAR
    18032 jps
    18005 jstat

\end{lstlisting}


\begin{lstlisting}[language=cshell]

jps -m remote.domain:2002

       3002 /opt/jdk1.7.0/demo/jfc/Java2D/Java2Demo.JAR
       3102 sun.tools.jstatd.jstatd -p 2002

    \end{lstlisting}