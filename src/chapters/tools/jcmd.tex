\section{jcmd命令}
\label{chap:tools_jcmd}

Sends diagnostic command requests to a running Java Virtual Machine (JVM).

\begin{lstlisting}[language=cshell]

    jcmd [-l|-h|-help]

    jcmd pid|main-class PerfCounter.print

    jcmd pid|main-class -f filename

    jcmd pid|main-class command[ arguments]

\end{lstlisting}


jcmd工具向JVM发送诊断命令请求。必须在当前运行着JVM机器上执行,并且具有与JVM具有相同的user和group标识。

运行jcmd不带参数或者使用 \-l 选项时,相当于jps (\textcolor[rgb]{0,0,1}{\ref{chap:tools_jps}}),会输出java进程标识符。

如果将进程标识符(pid)或主类(main-class)作为第一个参数,则jcmd将诊断命令请求发送给具有指定标识符或发具有指定main-class名称的Java进程。

如果使用0作为进程标识符,则会将诊断命令请求发送给所有可用的Java进程。


* Perfcounter.print

打印指定Java进程可用的性能计数器。性能计数器的列表可能因Java进程而异。


* -f filename

从文件中读取诊断命令。文件中每个命令必须单独一行。以\# 开头的命令会被忽略。如果命令行中包含\textcolor{codepurple}{stop}关键字则结束命令行读取。

* command [arguments]

要发送到指定Java进程的命令。可以通过向该进程发送\textcolor{codepurple}{help}命令来获得给定进程的可用诊断命令列表。

如果参数中包含空格,则必须使用单引号或者双引号扩起来。注意使用转义字符($\backslash$)转义单/双引号。


\begin{lstlisting}[language=cshell]

   jcmd 15 help     # 15为java进程id

    #### output ####

    15:
    The following commands are available:
    JFR.stop
    JFR.start
    JFR.dump
    JFR.check
    VM.native_memory
    VM.check_commercial_features
    VM.unlock_commercial_features
    ManagementAgent.stop
    ManagementAgent.start_local
    ManagementAgent.start
    VM.classloader_stats
    GC.rotate_log
    Thread.print
    GC.class_stats
    GC.class_histogram
    GC.heap_dump
    GC.finalizer_info
    GC.heap_info
    GC.run_finalization
    GC.run
    VM.uptime
    VM.dynlibs
    VM.flags
    VM.system_properties
    VM.command_line
    VM.version
    help

\end{lstlisting}

\begin{lstlisting}[language=cshell]
    jcmd 15 help GC.heap_info
\end{lstlisting}






