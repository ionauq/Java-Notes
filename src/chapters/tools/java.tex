\section{java命令}
\label{chap:tools_java}

Launches a Java application.

\begin{lstlisting}[language=cshell]

    java [options] classname [args]

    java [options] -jar filename [args]

\end{lstlisting}

通过启动JRE来调用指定的类,调用此类的main()方法。
\begin{lstlisting}[language=Java]
    public static void main(String[] args)
\end{lstlisting}

\subsection{选项} 


\begin{itemize}
    \item   标准选项
    \item   非标准选项 
    \item   高级运行时选项      (Advanced Runtime Options)
    \item   高级JIT编译器选项   (Advanced JIT Compiler Options)
    \item   高级可维护性选项    (Advanced Serviceability Options)
    \item   高级垃圾收集选项    (Advanced Garbage Collection Options)
\end{itemize}



\subsubsection{标准选项} 

Java虚拟机(JVM)的所有实现都保证支持的标准选项。


* -agentlib:libname[=options]

加载指定的本机代理库。在库名之后,可以使用特定于库的以逗号分隔的选项列表。

如果指定 -agentlib:foo 选项,则JVM尝试加载位于由系统环境变量名为LD\_LIBRARY\_PATH(在OS X系统下变量名为 DYLD\_LIBRARY\_PATH)指定位置下的libfoo.so的库。

下面的示例将展示如何加载堆分析工具(HPROF)库,并且获取堆栈深度为3,每20msCPU的简单采样信息:

\begin{lstlisting}[language=cshell]

    -agentlib:hprof=cpu=samples,interval=20,depth=3

\end{lstlisting}  

下面这个示例将展示如何加载Java调试线协议库并且监听8000端口的套接字连接,在主类加载之前挂起JVM:


\begin{lstlisting}[language=cshell]

    -agentlib:jdwp=transport=dt_socket,server=y,address=8000

\end{lstlisting} 





\-XX:+PrintStringTableStatistics


